\documentclass{article}
\usepackage{physics}
\usepackage{amsmath}

\title{Computatonal physics - Statistical Physics}
\author{Martin Johnsrud}

\begin{document}
    \maketitle
    
    \section*{Introduction}
    By numerically integrating Newtons equation of motion og many particles, confined to a box and interacting via the Leonard-Jones potential, it is possible to see the emergence of statistical behvior like the Maxwell-Boltzmann distribution. In this exercice, we explor this behavior, and ... (more to come)

    \section*{Single particel}
    A single particel with position $\vec x = (x_1, x_2)$ in a box is modeled in a potential 
    \begin{equation*}
        V_w(\vec x) = 
        \begin{cases}
            \frac{1}{2}K(r - R)^2, & r > R \\
            0, & r < R,
        \end{cases}
    \end{equation*}
    where $r = |\vec x|$. This leads to a force 
    \begin{equation*}
        \vec F_w(\vec x) = -\nabla V_w(\vec x) = 
        \begin{cases}
            -K(r - R)\hat x, & r>R \\
            0, & r<R.
        \end{cases}
    \end{equation*}

    \section*{$N$ particles}
    When moddeling $N$ particles $\{ \vec x_k\} = \{ (x_1^{(k)}, x_2^{(k)}) \}$, each one of them is subject to the force from the potential $V_w(\vec x_k)$, as well as a modified Leonard-Jones potential the interaction potential. The potential felt by particle $k$ is then
    \begin{equation*}
        V_k(\vec x_j) = 
        \sum_{r_{kj}<a}\epsilon \bigg[ \bigg( \frac{a}{r_{kj}}\bigg)^{12} - 2\bigg(\frac{a}{r_{kj}}\bigg)^{6} + 1 \bigg]
    \end{equation*}
    Here, $r_{kj} = |\vec x_k - \vec x_j|$. The force on particel $k$ by this potential is
    \begin{equation*}
        F_k (\vec x_j) = -\nabla_k V(\vec x_j) = \sum_{r_{kj}<a} 12 \epsilon \bigg[ \bigg( \frac{a}{r_{kj}}\bigg)^{12} - \bigg(\frac{a}{r_{kj}}\bigg)^{6}\bigg] \frac{\hat x_{kj}}{r_{kj}},
    \end{equation*}
    where $\hat x_{kj} = (\vec x_k - \vec x_j) / r_{kj}$
\end{document}